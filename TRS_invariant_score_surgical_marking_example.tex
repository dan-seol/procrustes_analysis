\PassOptionsToPackage{unicode=true}{hyperref} % options for packages loaded elsewhere
\PassOptionsToPackage{hyphens}{url}
%
\documentclass[
]{article}
\usepackage{lmodern}
\usepackage{amssymb,amsmath}
\usepackage{ifxetex,ifluatex}
\ifnum 0\ifxetex 1\fi\ifluatex 1\fi=0 % if pdftex
  \usepackage[T1]{fontenc}
  \usepackage[utf8]{inputenc}
  \usepackage{textcomp} % provides euro and other symbols
\else % if luatex or xelatex
  \usepackage{unicode-math}
  \defaultfontfeatures{Scale=MatchLowercase}
  \defaultfontfeatures[\rmfamily]{Ligatures=TeX,Scale=1}
\fi
% use upquote if available, for straight quotes in verbatim environments
\IfFileExists{upquote.sty}{\usepackage{upquote}}{}
\IfFileExists{microtype.sty}{% use microtype if available
  \usepackage[]{microtype}
  \UseMicrotypeSet[protrusion]{basicmath} % disable protrusion for tt fonts
}{}
\makeatletter
\@ifundefined{KOMAClassName}{% if non-KOMA class
  \IfFileExists{parskip.sty}{%
    \usepackage{parskip}
  }{% else
    \setlength{\parindent}{0pt}
    \setlength{\parskip}{6pt plus 2pt minus 1pt}}
}{% if KOMA class
  \KOMAoptions{parskip=half}}
\makeatother
\usepackage{xcolor}
\IfFileExists{xurl.sty}{\usepackage{xurl}}{} % add URL line breaks if available
\IfFileExists{bookmark.sty}{\usepackage{bookmark}}{\usepackage{hyperref}}
\hypersetup{
  pdftitle={A simple algorithm to get the TRS-invariant shape different between surgical markings},
  pdfauthor={Yunheum Dan Seol},
  pdfborder={0 0 0},
  breaklinks=true}
\urlstyle{same}  % don't use monospace font for urls
\usepackage[margin=1in]{geometry}
\usepackage{color}
\usepackage{fancyvrb}
\newcommand{\VerbBar}{|}
\newcommand{\VERB}{\Verb[commandchars=\\\{\}]}
\DefineVerbatimEnvironment{Highlighting}{Verbatim}{commandchars=\\\{\}}
% Add ',fontsize=\small' for more characters per line
\usepackage{framed}
\definecolor{shadecolor}{RGB}{248,248,248}
\newenvironment{Shaded}{\begin{snugshade}}{\end{snugshade}}
\newcommand{\AlertTok}[1]{\textcolor[rgb]{0.94,0.16,0.16}{#1}}
\newcommand{\AnnotationTok}[1]{\textcolor[rgb]{0.56,0.35,0.01}{\textbf{\textit{#1}}}}
\newcommand{\AttributeTok}[1]{\textcolor[rgb]{0.77,0.63,0.00}{#1}}
\newcommand{\BaseNTok}[1]{\textcolor[rgb]{0.00,0.00,0.81}{#1}}
\newcommand{\BuiltInTok}[1]{#1}
\newcommand{\CharTok}[1]{\textcolor[rgb]{0.31,0.60,0.02}{#1}}
\newcommand{\CommentTok}[1]{\textcolor[rgb]{0.56,0.35,0.01}{\textit{#1}}}
\newcommand{\CommentVarTok}[1]{\textcolor[rgb]{0.56,0.35,0.01}{\textbf{\textit{#1}}}}
\newcommand{\ConstantTok}[1]{\textcolor[rgb]{0.00,0.00,0.00}{#1}}
\newcommand{\ControlFlowTok}[1]{\textcolor[rgb]{0.13,0.29,0.53}{\textbf{#1}}}
\newcommand{\DataTypeTok}[1]{\textcolor[rgb]{0.13,0.29,0.53}{#1}}
\newcommand{\DecValTok}[1]{\textcolor[rgb]{0.00,0.00,0.81}{#1}}
\newcommand{\DocumentationTok}[1]{\textcolor[rgb]{0.56,0.35,0.01}{\textbf{\textit{#1}}}}
\newcommand{\ErrorTok}[1]{\textcolor[rgb]{0.64,0.00,0.00}{\textbf{#1}}}
\newcommand{\ExtensionTok}[1]{#1}
\newcommand{\FloatTok}[1]{\textcolor[rgb]{0.00,0.00,0.81}{#1}}
\newcommand{\FunctionTok}[1]{\textcolor[rgb]{0.00,0.00,0.00}{#1}}
\newcommand{\ImportTok}[1]{#1}
\newcommand{\InformationTok}[1]{\textcolor[rgb]{0.56,0.35,0.01}{\textbf{\textit{#1}}}}
\newcommand{\KeywordTok}[1]{\textcolor[rgb]{0.13,0.29,0.53}{\textbf{#1}}}
\newcommand{\NormalTok}[1]{#1}
\newcommand{\OperatorTok}[1]{\textcolor[rgb]{0.81,0.36,0.00}{\textbf{#1}}}
\newcommand{\OtherTok}[1]{\textcolor[rgb]{0.56,0.35,0.01}{#1}}
\newcommand{\PreprocessorTok}[1]{\textcolor[rgb]{0.56,0.35,0.01}{\textit{#1}}}
\newcommand{\RegionMarkerTok}[1]{#1}
\newcommand{\SpecialCharTok}[1]{\textcolor[rgb]{0.00,0.00,0.00}{#1}}
\newcommand{\SpecialStringTok}[1]{\textcolor[rgb]{0.31,0.60,0.02}{#1}}
\newcommand{\StringTok}[1]{\textcolor[rgb]{0.31,0.60,0.02}{#1}}
\newcommand{\VariableTok}[1]{\textcolor[rgb]{0.00,0.00,0.00}{#1}}
\newcommand{\VerbatimStringTok}[1]{\textcolor[rgb]{0.31,0.60,0.02}{#1}}
\newcommand{\WarningTok}[1]{\textcolor[rgb]{0.56,0.35,0.01}{\textbf{\textit{#1}}}}
\usepackage{graphicx,grffile}
\makeatletter
\def\maxwidth{\ifdim\Gin@nat@width>\linewidth\linewidth\else\Gin@nat@width\fi}
\def\maxheight{\ifdim\Gin@nat@height>\textheight\textheight\else\Gin@nat@height\fi}
\makeatother
% Scale images if necessary, so that they will not overflow the page
% margins by default, and it is still possible to overwrite the defaults
% using explicit options in \includegraphics[width, height, ...]{}
\setkeys{Gin}{width=\maxwidth,height=\maxheight,keepaspectratio}
\setlength{\emergencystretch}{3em}  % prevent overfull lines
\providecommand{\tightlist}{%
  \setlength{\itemsep}{0pt}\setlength{\parskip}{0pt}}
\setcounter{secnumdepth}{-2}
% Redefines (sub)paragraphs to behave more like sections
\ifx\paragraph\undefined\else
  \let\oldparagraph\paragraph
  \renewcommand{\paragraph}[1]{\oldparagraph{#1}\mbox{}}
\fi
\ifx\subparagraph\undefined\else
  \let\oldsubparagraph\subparagraph
  \renewcommand{\subparagraph}[1]{\oldsubparagraph{#1}\mbox{}}
\fi

% set default figure placement to htbp
\makeatletter
\def\fps@figure{htbp}
\makeatother


\title{A simple algorithm to get the TRS-invariant shape different between
surgical markings}
\author{Yunheum Dan Seol}
\date{5/26/2020}

\begin{document}
\maketitle

\begin{Shaded}
\begin{Highlighting}[]
\CommentTok{#install all the R packages.}
\CommentTok{#if the packages are not installed, please uncomment the line below and run it;}
\CommentTok{#install.packages(c("BiocManager", "magrittr", "shapes", "Rcpp", "RcppArmadillo", "kmlShape"))}
\end{Highlighting}
\end{Shaded}

\begin{Shaded}
\begin{Highlighting}[]
\CommentTok{#Import all the required R packages.}
\KeywordTok{library}\NormalTok{(compiler)}
\KeywordTok{library}\NormalTok{(magrittr)}
\KeywordTok{library}\NormalTok{(EBImage)}
\KeywordTok{library}\NormalTok{(shapes)}
\end{Highlighting}
\end{Shaded}

\begin{verbatim}
## 
## Attaching package: 'shapes'
\end{verbatim}

\begin{verbatim}
## The following object is masked from 'package:EBImage':
## 
##     abind
\end{verbatim}

\begin{verbatim}
## The following objects are masked from 'package:magrittr':
## 
##     add, mod
\end{verbatim}

\begin{Shaded}
\begin{Highlighting}[]
\KeywordTok{library}\NormalTok{(kmlShape)}
\end{Highlighting}
\end{Shaded}

\begin{verbatim}
## Loading required package: class
\end{verbatim}

\begin{verbatim}
## Loading required package: longitudinalData
\end{verbatim}

\begin{verbatim}
## Loading required package: clv
\end{verbatim}

\begin{verbatim}
## Loading required package: cluster
\end{verbatim}

\begin{verbatim}
## Loading required package: rgl
\end{verbatim}

\begin{verbatim}
## Loading required package: misc3d
\end{verbatim}

\begin{verbatim}
## Loading required package: kml
\end{verbatim}

\begin{verbatim}
## Loading required package: lattice
\end{verbatim}

\begin{verbatim}
## 
## Attaching package: 'kmlShape'
\end{verbatim}

\begin{verbatim}
## The following object is masked from 'package:longitudinalData':
## 
##     distFrechet
\end{verbatim}

\begin{Shaded}
\begin{Highlighting}[]
\KeywordTok{library}\NormalTok{(Rcpp)}
\KeywordTok{library}\NormalTok{(RcppArmadillo)}
\CommentTok{#these are helper modules written in RcppArmadillo for faster computation}
\KeywordTok{sourceCpp}\NormalTok{(}\StringTok{"imageThinning.cpp"}\NormalTok{)}
\KeywordTok{sourceCpp}\NormalTok{(}\StringTok{"downSamplePoints.cpp"}\NormalTok{)}
\end{Highlighting}
\end{Shaded}

\begin{Shaded}
\begin{Highlighting}[]
\CommentTok{#testing functions in imageThinning.cpp}
\NormalTok{m1 =}\StringTok{ }\KeywordTok{matrix}\NormalTok{(}\KeywordTok{c}\NormalTok{(}\DecValTok{1}\NormalTok{,}\DecValTok{2}\NormalTok{,}\DecValTok{3}\NormalTok{,}\DecValTok{4}\NormalTok{), }\DataTypeTok{nrow=}\DecValTok{2}\NormalTok{)}
\NormalTok{m2 =}\StringTok{ }\KeywordTok{matrix}\NormalTok{(}\KeywordTok{c}\NormalTok{(}\DecValTok{1}\NormalTok{,}\DecValTok{2}\NormalTok{,}\DecValTok{2}\NormalTok{,}\DecValTok{4}\NormalTok{), }\DataTypeTok{nrow=}\DecValTok{2}\NormalTok{)}
\NormalTok{diff =}\StringTok{ }\KeywordTok{abs_diff}\NormalTok{(m1, m2)}
\KeywordTok{countNonZero}\NormalTok{(diff)}
\end{Highlighting}
\end{Shaded}

\begin{verbatim}
## [1] 1
\end{verbatim}

\begin{Shaded}
\begin{Highlighting}[]
\CommentTok{# we read in the images}
\NormalTok{bilobe1 <-}\StringTok{ }\KeywordTok{readImage}\NormalTok{(}\StringTok{'bilobe1_rgb.png'}\NormalTok{)}
\NormalTok{bilobe2 <-}\StringTok{ }\KeywordTok{readImage}\NormalTok{(}\StringTok{'bilobe2_rgb.png'}\NormalTok{)}
\CommentTok{#bilobe1}
\end{Highlighting}
\end{Shaded}

\begin{Shaded}
\begin{Highlighting}[]
\CommentTok{#turn images into grayscale}
\NormalTok{binary_bilobe1 <-}\StringTok{ }\NormalTok{bilobe1 }\OperatorTok\StringTok{ }\KeywordTok{channel}\NormalTok{(., }\StringTok{"gray"}\NormalTok{) }\OperatorTok\StringTok{ `}\DataTypeTok{<}\StringTok{`}\NormalTok{(., }\FloatTok{0.5}\NormalTok{) }\OperatorTok\StringTok{ }\KeywordTok{bwlabel}\NormalTok{(.)}
\NormalTok{binary_bilobe2 <-}\StringTok{ }\NormalTok{bilobe2 }\OperatorTok\StringTok{ }\KeywordTok{channel}\NormalTok{(., }\StringTok{"gray"}\NormalTok{) }\OperatorTok\StringTok{ `}\DataTypeTok{<}\StringTok{`}\NormalTok{(., }\FloatTok{0.5}\NormalTok{) }\OperatorTok\StringTok{ }\KeywordTok{bwlabel}\NormalTok{(.)}
\end{Highlighting}
\end{Shaded}

\begin{Shaded}
\begin{Highlighting}[]
\KeywordTok{display}\NormalTok{(binary_bilobe1)}
\end{Highlighting}
\end{Shaded}

\includegraphics{TRS_invariant_score_surgical_marking_example_files/figure-latex/unnamed-chunk-6-1.pdf}

\begin{Shaded}
\begin{Highlighting}[]
\KeywordTok{display}\NormalTok{(binary_bilobe2)}
\end{Highlighting}
\end{Shaded}

\includegraphics{TRS_invariant_score_surgical_marking_example_files/figure-latex/unnamed-chunk-7-1.pdf}

\begin{Shaded}
\begin{Highlighting}[]
\CommentTok{#thin the images}
\NormalTok{thinned1 <-}\StringTok{ }\NormalTok{binary_bilobe1 }\OperatorTok\StringTok{ }\NormalTok{thinImage}
\KeywordTok{display}\NormalTok{(thinned1)}
\end{Highlighting}
\end{Shaded}

\includegraphics{TRS_invariant_score_surgical_marking_example_files/figure-latex/unnamed-chunk-8-1.pdf}

\begin{Shaded}
\begin{Highlighting}[]
\NormalTok{thinned2 <-}\StringTok{ }\NormalTok{binary_bilobe2 }\OperatorTok\StringTok{ }\NormalTok{thinImage}
\KeywordTok{display}\NormalTok{(thinned2)}
\end{Highlighting}
\end{Shaded}

\includegraphics{TRS_invariant_score_surgical_marking_example_files/figure-latex/unnamed-chunk-9-1.pdf}

\begin{Shaded}
\begin{Highlighting}[]
\KeywordTok{writeImage}\NormalTok{(thinned1, }\StringTok{"thinned_bilobe1.png"}\NormalTok{)}
\KeywordTok{writeImage}\NormalTok{(thinned2, }\StringTok{"thinned_bilobe2.png"}\NormalTok{)}
\end{Highlighting}
\end{Shaded}

\begin{Shaded}
\begin{Highlighting}[]
\CommentTok{#print(dim(getCurvePoints(thinned1))) # 1037 2}
\CommentTok{#print(dim(getCurvePoints(thinned2))) # 1751 2}
\NormalTok{curvePoints1 <-}\StringTok{ }\KeywordTok{getCurvePoints}\NormalTok{(thinned1)}
\NormalTok{curvePoints2 <-}\StringTok{ }\KeywordTok{getCurvePoints}\NormalTok{(thinned2)}
\NormalTok{dim1 <-}\StringTok{ }\KeywordTok{dim}\NormalTok{(curvePoints1)[}\DecValTok{1}\NormalTok{]}
\NormalTok{dim2 <-}\StringTok{ }\KeywordTok{dim}\NormalTok{(curvePoints2)[}\DecValTok{1}\NormalTok{]}
\NormalTok{minDim <-}\StringTok{ }\KeywordTok{min}\NormalTok{(dim1, dim2)}
\NormalTok{updatedCurvePoints1 <-}\StringTok{ }\KeywordTok{DouglasPeuckerNbPoints}\NormalTok{(curvePoints1}\OperatorTok{$}\NormalTok{x_coords, curvePoints1}\OperatorTok{$}\NormalTok{y_coords, minDim)}
\end{Highlighting}
\end{Shaded}

\begin{verbatim}
## Warning in DouglasPeuckerNbPoints(curvePoints1$x_coords,
## curvePoints1$y_coords, : [DouglasPeukerNbPoints] the simplified curve perfectly
## fit with the original with only 507 points
\end{verbatim}

\begin{Shaded}
\begin{Highlighting}[]
\NormalTok{updatedCurvePoints2 <-}\StringTok{ }\KeywordTok{DouglasPeuckerNbPoints}\NormalTok{(curvePoints2}\OperatorTok{$}\NormalTok{x_coords, curvePoints2}\OperatorTok{$}\NormalTok{y_coords, minDim)}
\end{Highlighting}
\end{Shaded}

\begin{verbatim}
## Warning in DouglasPeuckerNbPoints(curvePoints2$x_coords,
## curvePoints2$y_coords, : [DouglasPeukerNbPoints] the simplified curve perfectly
## fit with the original with only 742 points
\end{verbatim}

\begin{Shaded}
\begin{Highlighting}[]
\NormalTok{arr1 <-}\StringTok{ }\KeywordTok{as.matrix}\NormalTok{(updatedCurvePoints1)}
\NormalTok{arr2 <-}\StringTok{ }\KeywordTok{as.matrix}\NormalTok{(updatedCurvePoints2)}
\NormalTok{samples_layered <-}\StringTok{ }\KeywordTok{array}\NormalTok{(}\KeywordTok{c}\NormalTok{(arr1, arr2), }\DataTypeTok{dim=}\KeywordTok{c}\NormalTok{(minDim, }\DecValTok{2}\NormalTok{, }\DecValTok{2}\NormalTok{))}
\end{Highlighting}
\end{Shaded}

\begin{Shaded}
\begin{Highlighting}[]
\NormalTok{TRSD <-}\StringTok{ }\KeywordTok{list}\NormalTok{()}
\NormalTok{procGPAcompiled <-}\StringTok{ }\KeywordTok{cmpfun}\NormalTok{(procGPA)}
\NormalTok{transformed <-}\StringTok{ }\KeywordTok{procGPAcompiled}\NormalTok{(samples_layered, }\DataTypeTok{reflect=}\OtherTok{FALSE}\NormalTok{)}
\NormalTok{TRS <-}\StringTok{ }\KeywordTok{transformations}\NormalTok{(transformed}\OperatorTok{$}\NormalTok{rotated, samples_layered)}

\NormalTok{TRSD[}\StringTok{"T"}\NormalTok{] <-}\StringTok{ }\NormalTok{TRS}\OperatorTok{$}\NormalTok{translation}
\end{Highlighting}
\end{Shaded}

\begin{verbatim}
## Warning in TRSD["T"] <- TRS$translation: number of items to replace is not a
## multiple of replacement length
\end{verbatim}

\begin{Shaded}
\begin{Highlighting}[]
\NormalTok{TRSD[}\StringTok{"R"}\NormalTok{] <-}\StringTok{ }\NormalTok{TRS}\OperatorTok{$}\NormalTok{rotation}
\end{Highlighting}
\end{Shaded}

\begin{verbatim}
## Warning in TRSD["R"] <- TRS$rotation: number of items to replace is not a
## multiple of replacement length
\end{verbatim}

\begin{Shaded}
\begin{Highlighting}[]
\NormalTok{TRSD[}\StringTok{"S"}\NormalTok{] <-}\StringTok{ }\NormalTok{TRS}\OperatorTok{$}\NormalTok{scale}
\end{Highlighting}
\end{Shaded}

\begin{verbatim}
## Warning in TRSD["S"] <- TRS$scale: number of items to replace is not a multiple
## of replacement length
\end{verbatim}

\begin{Shaded}
\begin{Highlighting}[]
\NormalTok{TRSD[}\StringTok{"D"}\NormalTok{] <-}\StringTok{ }\NormalTok{transformed}\OperatorTok{$}\NormalTok{rho}
\end{Highlighting}
\end{Shaded}

\begin{verbatim}
## Warning in TRSD["D"] <- transformed$rho: number of items to replace is not a
## multiple of replacement length
\end{verbatim}

\begin{Shaded}
\begin{Highlighting}[]
\NormalTok{TRSD}
\end{Highlighting}
\end{Shaded}

\begin{verbatim}
## $T
## [1] -825.0752
## 
## $R
## [1] -0.5415227
## 
## $S
## [1] 1.038342
## 
## $D
## [1] 0.5002801
\end{verbatim}

\end{document}
